% IIS — Information and Measurement System
% Core mathematical formulation (conceptual engineering model)

\documentclass[12pt]{article}
\usepackage[a4paper,margin=2.5cm]{geometry}
\usepackage{amsmath}
\usepackage{amssymb}

\begin{document}

\section*{IIS Core Model}

\subsection*{Inputs}
Let the normalized input parameters be:
\[
A, \Gamma, H, V \in [0, 1]
\]
where:
\begin{itemize}
  \item $A$ --- activity-related parameter,
  \item $\Gamma$ --- stress-related parameter,
  \item $H$ --- stability parameter,
  \item $V$ --- variability parameter.
\end{itemize}

\subsection*{Weights}
Let the model weights be:
\[
w_A = 0.35,\quad w_\Gamma = 0.25,\quad w_H = 0.30,\quad w_V = 0.10
\]
with the normalization constraint:
\[
w_A + w_\Gamma + w_H + w_V = 1
\]

\subsection*{Linear aggregation}
Define the raw aggregated value:
\[
\mathrm{IIS}_{raw} = w_A A + w_\Gamma \Gamma + w_H H + w_V V
\]
Explicitly:
\[
\mathrm{IIS}_{raw} = 0.35A + 0.25\Gamma + 0.30H + 0.10V
\]

\subsection*{Non-linear mapping (sigmoid)}
Define the sigmoid function:
\[
\sigma(x) = \frac{1}{1 + e^{-x}}
\]

The IIS output index is:
\[
\mathrm{IIS} = \sigma(\mathrm{IIS}_{raw})
\]

\subsection*{Output range}
Since $\sigma(x) \in (0, 1)$ for all real $x$, it follows that:
\[
\mathrm{IIS} \in (0, 1)
\]

\subsection*{Note}
This model is an analytical aggregation framework.
It does not define measurement methods and must be interpreted only
within documented assumptions and limitations.

\end{document}
